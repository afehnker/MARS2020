\documentclass{article}
\usepackage{a4wide}
\usepackage{amsmath,amssymb,graphicx}
\usepackage{color}
\usepackage{hyperref}

\pagestyle{empty}



\begin{document}
\begin{center}
{\Large\bf Workshop Proposal}\\[5mm]
{\large 4th Workshop on }\\[2mm]
{\LARGE\bf Models for Formal Analysis of Real Systems}\\[2mm]
{\LARGE MARS 2020}
\end{center}

\paragraph*{Organisation.} MARS 2020 is intended to be a one-day event
held in conjunction with the European Joint Conferences on Theory and
Practice of Software (ETAPS) in 2020.

The organisers are
\begin{itemize}
\item[] \href{https://wwwhome.ewi.utwente.nl/~fehnkera/}{Ansgar Fehnker},
  University of Twente, the Netherlands.
\item[] \href{http://convecs.inria.fr/people/Hubert.Garavel/}{Hubert Garavel},
  INRIA, France .
\end{itemize}
The Programme Committee will include about 10 additional experts in
the area of this workshop.

All submissions will be peer-reviewed by at least three referees based
on their novelty, relevance and technical merit.  Contributions are
limited to $12$ pages EPTCS style (not counting the appendix), but
shorter extended abstracts will be welcome.  We plan to allow
appendices of arbitrary length, so as to enable the inclusion of all
details of a formalised model (see below).

We plan to publish the accepted papers in EPTCS, as for the previous editions of this workshop.

\paragraph*{Invited Speaker.} Funding will be sought in order to have an invited speaker.
Should funding not be obtained, we could consider someone who is participating
in ETAPS anyway.

\paragraph*{Overall Aims and Goals.}

Specification formalisms and modelling techniques have often been
developed with formal analysis and formal verification in mind. To
show applicability, toy examples or tiny case studies are typically
presented in research papers. When the theory needs to be developed, this approach is reasonable.  However, to show that a developed
approach actually scales to real systems, large case studies are
essential.

The development of formal models of real systems usually requires a
perfect understanding of informal descriptions of systems which are usually just
written in English. Based on the type of system, an adequate
specification formalism needs to be chosen, and the informal
specification translated into it. Abstraction from unimportant details
then yields an accurate, formal model of the real system.  The process
of developing a detailed and accurate model usually takes a large
amount of time, often months or years; without even starting a formal
analysis.

When publishing the results on a formal analysis in a
scientific paper, details of the model have to be skipped due to lack
of space, and often the lessons learnt from modelling are not
discussed since they are not within the main focus of the paper.

The MARS workshop aims at discussing exactly these unmentioned lessons,
emphasising modelling over verification.
Examples are:
\begin{itemize}
\setlength\itemsep{0pt}
\setlength\leftmargin{10pt}
   \item Which formalism was chosen, and why?
     \item Which abstractions have been made, and why?
      \item How were important characteristics of the system modelled?
    
    \item Were there any complications while modelling the system?
    \item Which measures were taken to guarantee the accuracy of the model?
\end{itemize}

We thus invite papers that present full models of real systems, which may
lay the basis for future comparison and analysis. The MARS workshop intends
to bring together researchers from different communities who are developing
formal models of real systems, especially in areas where large models occur,
such as networks, cyber-physical systems, or hardware/software codesign. An
aim of the workshop is to present different modelling approaches and discuss
the pros and cons of each of them.

In addition to the EPTCS proceedings, the formal models presented during
the workshop will be archived in the MARS repository
(\url{http://mars-workshop.org/repository.html}), a growing, diverse collection of
realistic benchmarks, provided in machine-readable form and licensed under
Creative Commons. The existence of the MARS repository is a unique feature
that makes MARS papers available to the wider community so that others can
reproduce experiments, perform further analyses, and try the same case studies
with different formal methods.

\paragraph*{Relevance to the ETAPS community.}

Large models and realistic cases are a notorious and recurring concern
for several ETAPS member conferences, such as TACAS, FASE, or ESOP, as
well as several ETAPS affiliated workshops.

When presenting formal models of real systems, the workshop is meant
to provide a forum for the discussion of features and problems, as
well as shortcomings and benefits of certain modelling and
verification mechanisms.  As a consequence, the workshop is expected to
open new directions for extending existing approaches.


\paragraph{Web address of the workshop}
\url{http://mars-workshop.org/mars2020/}

\paragraph*{Previous Events.}
The workshop is the fourth in the MARS series of workshop.
 The first MARS workshop has
been organised in 2015 at LPAR in Suva, Fiji.
\url{http://mars-workshop.org/mars2015/}. 
The second MARS workshop was be held as an affiliated workshop of
ETAPS 2017 in Uppsala, Sweden. 
\url{http://mars-workshop.org/mars2017/}.
The third MARS workshop was be held as an affiliated workshop of
ETAPS 2018 in Thessaloniki, Greece. 
\url{http://mars-workshop.org/mars2018/}.

The proceedings to all three events appeared in EPTCS. In addition to the proceedings, the MARS repository keeps track of the
models submitted during former editions of the MARS workshops.

\paragraph{Contact information of the workshop organisers:}
\begin{description}
\item{Ansgar Fehnker}\\ University of Twente\\
P.O. Box 217\\
7500 AE Enschede\\
The Netherlands

\item{Hubert Garavel}\\
INRIA Grenoble Rh\^{o}ne-Alpes\\ 
655, avenue de l'Europe \\
F-38330 Montbonnot Saint-Martin\\ 
FRANCE 
\end{description}
    \paragraph{An estimate of the audience size:}
25.
    \paragraph{Proposed format of the workshop:}
% (for example, regular talks, tool demos, poster presentations, etc.)
Talks.
    \paragraph{Procedures for selecting papers and participants:}
    All submissions will be peer-reviewed by at least three referees.
    Selection is based on novelty, relevance and technical merit.
    \paragraph{Programme Committee:} At the time of writing the PC has not yet been selected. Once it is known, the composition of the PC will be announced on our website.
    \paragraph{Special technical or AV needs:}
None.

\end{document}

